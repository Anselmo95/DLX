%%%%%%%%%%%%%%%%%%%%%%%%%%%%%%%%%%%%%%%%%%%%%%%%%%%%
%\graphicspath{chapters/figures/}
\chapter{Synthesis and Place\&Route}
\label{chap_synth}

%%%%%%%%%%%%%%%%%%%%%%%%%%%%%%%%%%%%%%%%%%%%%%%%%%%%%%%%%%%
\section{Synthesis}
After the design phase, we used the Synopsis DC Compiler tool to produce a synthesized version of our \dlx processor. We decided to run the script, provided in appendix \ref{syn_scr} for the synthesis several times, and we obtained quite different results.
At first, we have compiled all the needed components respecting the existing hierarchies. It is worth nothing that we have provided for every component also the configuration, in order to have the possibility of trying different implementations for future improvement. At first the script for the synthesis just launches a medium effort compilation, so that we can get an overall idea of the slack that we have and the maximum clock frequency that we can reach. 
Then, reports for area, power and timing are generated.

After this step, in the first versions of the script, we were just launching again another compilation with an high effort. After many trials and researches, we found out that we can ask the tool to start from the previously obtained results and try to do a new mapping, targeting better results. The command we have used is:
\begin{lstlisting}[style=B]
	compile -map_effort high -incremental_mapping
\end{lstlisting}

We also noticed that, while trying to decrease the \textit{MAX\_DELAY}, the results obtained change in a relevant way (\SI{62}{\nano\second} of difference). In general, the second compilation is capable of sensibly improving the results obtained before, in almost all ways (area, power, timing). In table \ref{tab_rep} are shown the results related to the two stage of the compilation we finally obtained.
The final reports generated are reported in appendix \ref{syn_rpt}.

\begin{table}[]
	\centering
	\begin{tabular}{l|cc}
		\toprule
				& Medium  						& High  								\\
		\midrule
		Area	& \SI{24002.24}{\nano\meter^2}	&  	\SI{24071.93}{\nano\meter^2}		\\
		Power	& \SI{8.5142}{\milli\watt}		&  	\SI{8.5306}{\milli\watt}			\\
		Slack	& \SI{-0.46}{\nano\second}		&	\SI{0.0}{\nano\second}				\\ 
		\bottomrule
	\end{tabular}
\caption{Comparison for synthesis results}
\label{tab_rep}
\end{table}

\section{Place \& Route}

Using the \textit{Cadence} tool, we were finally able to physically place on the die the synthesized components obtained before. We modified the \textit{.globals} file, including the post synthesis file generated in the previous phase. Then, we used the tool as we did in the last laboratory, and we finally got the requested result. Several reports were generated as well, which are included in the \textsc{SYN} folder. Even after this phase, we noticed that some of the paths does not exactly respect the hold time required. A possible workaround could be to simply slightly increase the clock period, as well as remapping the entire \dlx and proceed in a new place \& route. We still decided not to proceed, since all the three violations are related to signals which are supposed to deal with the external memory as it can be seen in \ref{pr_rpt}, whose performances are unknown. In particular, the analysis could lead to a wrong estimation of the timing requirements. We decided not to proceed with further changes and stick with the current version.
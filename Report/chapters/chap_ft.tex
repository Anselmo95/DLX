%%%%%%%%%%%%%%%%%%%%%%%%%%%%%%%%%%%%%%%%%%%%%%%%%%%%
%\graphicspath{chapters/figures/}
\section{Fetch Stage}
\label{chap_ft}

%%%%%%%%%%%%%%%%%%%%%%%%%%%%%%%%%%%%%%%%%%%%%%%%%%%%%%%%%%%
\subsection{Introduction}
The fetch stage is important to provide to our processor a new instruction at every clock cycle, if needed. It is the first stage that a function actually enters in its execution flow.

Here are presented the signals involved:
\begin{itemize}
	\item \textit{CLK} is the usual clock signal needed to synchronize the system
	\item \textit{STALL} is a control signal, that blocks the update of the \textsf{PC} and therefore does not let any other operations to be issued
	\item \textit{RST} it's the canonical reset signal
	\item \textit{RST\_DEC} \textbf{ALBI PLEASE ADD EXPLANATION HERE}
	\item \textit{PC\_SEL} is a control signal which selects the next value of the \textit{PC}, either from the normal execution flow or from the ALU
	\item \textit{JB\_INST} is the computed new \textit{PC} value coming from the \alu
	\item \textit{FUNC} is a part of the instruction, for the \textsc{R\_TYPE} operations
	\item \textit{OPCODE} represents the type of operation in the instruction set to be performed by the processor
	\item \textit{NPC} which is the canonical updated \pc, pointing to the next instruction
	\item \textit{INST\_OUT} is a signal that comes from the IRAM, after the instruction has been read
	\item \textit{MISS\_HIT} is used to inform the system whether the prediction made was correct or not
	\item \textit{FLUSH\_CTL} is a control signal which allows us to clean the pipe in case of a wrong prediction
\end{itemize}

